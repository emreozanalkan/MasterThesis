% Chapter 5

\chapter{Conclusions} % Main chapter title
\label{Chapter5} % For referencing the chapter elsewhere, use \ref{Chapter3} 

\lhead{Chapter 5. \emph{Conclusions}} % This is for the header on each page - perhaps a shortened title
%-----------------------------------------------------------------------------
%\section{Conclusions}
\label{section:Conclusions}
	In this thesis, a voting scheme approach is proposed to address the problem of activity recognition in an operating room using a multi-view RGBD camera system. The 4D spatio-temporal space is divided into smaller local patches using a data-driven non-rigid layout by \cite{twinanda2015data} which also helps with sparse interest points. A two-level classification strategy is introduced, i.e., learning the probability votes and learning the weights. Furthermore, two different classification approaches are compared using two-level classification: one-model and multi-model. The one-model approach trained a single classifier model with all patches from the all video clips in each level, however, the multi-model approach trained separate classifier models for  each patch type in a video clip in the fist level. Finally, the proposed voting strategy collects votes from the each local patch from a video clip and the weights are used to recognize the activity. The voting scheme is evaluated on a new dataset from \cite{twinanda2015data} which consists of annotated 1734 real surgical video clips with 15 different surgical activity types. In order to make comprehensive comparison, we compared the proposed voting scheme with non-voting scheme approach from \cite{twinanda2015data}. We have shown that the proposed voting scheme gives promising results. 
    
    The current spatio-temporal interest point detection is based on movements occur in the video clips where the detection is in the borders of the action. Hence, the detected interest points are sparse and ignore the information from stationary parts of the videos. Additionally, the bag-of-words approach is another limit which encodes the features into sparse representation. It would be interesting to incorporate extraction of dense features and different feature encoding approaches. Since the voting scheme uses local parts of 4D spatio-temporal space, it would be interesting to use the voting scheme to recognize concurrent activities.
    
    
%     The sparsity of the features because of the bag-of-words approach is addressed by introducing two level classification: first the weights of the local patches are learnt and then the learnt weights are used for training. Furthermore, two different classification approaches are compared using two-level classification: one-model and multi-model. The one-model approach trained a single model with the all patches from the all video clips, however, the multi-model approach trained separate models for the each patch type in a video clip. Finally, the proposed voting strategy collects votes from the each local patch from a video clip and a majority vote is used to recognize the activity. The voting scheme is evaluated on a new dataset from \cite{twinanda2015data} which consists of annotated 1734 real surgical video clips with 15 different surgical activity types. In order to make comprehensive comparison, we compared the proposed voting scheme with non-voting scheme approach from \cite{twinanda2015data}. We have shown that the proposed voting scheme gives promising results and is open for developing new strategies, e.g., concurrent activity recognition. The current spatio-temporal interest point detection is based on movements occur in the video clips where the detection is in the borders of the action. Hence, the detected interest points are sparse and ignoring the information from stationary parts of the videos. Additionally, the bag-of-words approach is another limit which encodes the features into sparse representation. It would be interesting to incorporate extraction of dense features and different feature encoding approaches.